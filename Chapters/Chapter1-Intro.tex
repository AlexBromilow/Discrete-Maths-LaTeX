\chapter{Intro to Discrete Maths}

What is Discrete maths? I like to think of it as an introduction into mathematical thinking. We're going to learn how to think logically, mathematically, symbolically and creatively.

Discrete maths takes this set of skills and applies them to a kind of math problem specifically useful tot hsoe in Computer Science or Information Technology.

The term 'Discrete' puts this kind of maths into opposition with 'Continuous' maths you may have seen before. For example, the graph of the function $f(x) = x^2$, where $x$ can be any real number input (e.g. $1$, $\frac{1}{2}$,$\pi$, etc\dots). \emph{Discrete} values are seen as separate entities with nothing between them (e.g. 1,2,17\dots).

This is useful in computer sciences since this is how information is foten used in that context. For example entries in a database are \emph{discrete} entries, there aren't an infinite number of other entries between each entry.

This course is going to cover probability, counting, graph theory and a whole bunch of logic. To really get the most from this course, you will need to do more than just read through each chapter. These chapters (\emph{modules}) will give you the foundational scaffolding to go and try things out for yourself. You may also find it helpful to think of questions around the topics and then go and research to find the answers.

Good luck and, most of all, enjoy!

