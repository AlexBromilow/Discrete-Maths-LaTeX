\chapter{Logical Arguments}

\section{Introduction to Logical Arguments}
In logic, \emph{arguments} are a list of statements (called \emph{premises}) followed by a conclusion, usually of the form ``If $p$ and $q$, then $r$".

\begin{dfn}[label={def:validArguments}]{Valid Arguments}{dfnValidArguments}
    A \emph{valid argument} is an argument where the conclusion logically follows from the premises, regardless of the truth value of the premises.
\end{dfn}

\begin{exmpl}[label={exmpl:validArgument}]{A Valid Argument}{xmplValidArgument}
    Let's take a look at the form a valid argument would take. Our two premises are
    \begin{center}
        If I do the dishes, I will feel better\\
        I do the dishes
    \end{center}
    Our conclusion therefore is
    \begin{center}
        I feel better
    \end{center}
    This argument is \emph{valid} because, if our two premises are true, then the conclusion logically follows
\end{exmpl}

\begin{exmpl}[label={exmpl:invalidArgument}]{An Invalid Argument}{xmplInvalidArgument}
    Let's take a look what an \emph{invalid} argument would look like. Our premises are the same as before
    \begin{center}
        If I do the dishes, I will feel better\\
        I do the dishes
    \end{center}
    This time our conclusion is
    \begin{center}
        The sky is blue
    \end{center}
    In this case, the conclusion does not follow from the premises, nothing about us doing the dishes indicates that the sky is blue.

    Note that it doesn't matter if the conclusion is true or not, the argument is invalid because it conclusion doesn't follow from the premises.
\end{exmpl}

This particluar form of argument actually has a name

\begin{dfn}[label={def:modusPonens}]{Modus Ponens Argument}{dfnModusPonens}
    A \emph{Modus Ponens} argument is an argument of the form
    \begin{center}
        If $p$, then $q$.\\
        $p$.\\
        Therefore, $q$.
    \end{center}
\end{dfn}

Let's use a truth table to try and get a better appreciation for this style of argument

\begin{center}
    \begin{tabular}{|c|c|c|c|c|}
        \hline
        \multicolumn{2}{|c|}{Variables} & \multicolumn{2}{|c|}{Premises} & Conclusion             \\
        \hline
        $p$                             & $q$                            & $p \to q$  & $p$ & $q$ \\
        \hline
        T                               & T                              & T          & T   & T   \\
        \hline
        T                               & F                              & F          & T   & N.A \\
        \hline
        F                               & T                              & T          & F   & N.A \\
        \hline
        F                               & F                              & T          & F   & N.A \\
        \hline
    \end{tabular}
\end{center}

We have put ``N.A" in the final 3 rows of the last column for one very important reason - our argument is assuming that both premises are true and this only occurs in the top row so this is the only one we care about evaluating.\\

Another important form of a logical argument is essentially the contrapositive (\cref{def:contrapositive}) of the Modus Ponens argument

\begin{dfn}[label={def:modusTollens}]{Modus Tollens Argument}{dfnModusTollens}
    A \emph{Modus Tollens} argument is an argument of the form
    \begin{center}
        If $p$, then $q$.\\
        $\lnot q$.\\
        Therefore, $\lnot p$.
    \end{center}
\end{dfn}

Let's take a look at the truth table for this style of argument as well
\begin{center}
    \begin{tabular}{|c|c|c|c|c|}
        \hline
        \multicolumn{2}{|c|}{Variables} & \multicolumn{2}{|c|}{Premises} & Conclusion                         \\
        \hline
        $p$                             & $q$                            & $p \to q$  & $\lnot q$ & $\lnot p$ \\
        \hline
        T                               & T                              & T          & F         & N.A       \\
        \hline
        T                               & F                              & F          & T         & N.A       \\
        \hline
        F                               & T                              & T          & F         & N.A       \\
        \hline
        F                               & F                              & T          & T         & T         \\
        \hline
    \end{tabular}
\end{center}

Similarly to before, we only want to evaluate when both of our premises are true, in this case that is the final row. When both our premises are true, $p$ is false and therefore our conclusion, $\lnot p$ is true, as required.

Let's look at an example of a Modus Tollens argument

\begin{exmpl}[label={exmpl:modusTollens}]{Modus Tollens Argument}{xmplModusTollens}
    Let's evaluate the following argument
    \begin{center}
        If I am the POTUS then I'm an American citizen.\\
        I am not an American citizen.\\
        Therefore, I am not the POTUS.
    \end{center}
    Here we have $p$:``I am the POTUS" and $q$:``I'm an American citizen". This is an argument of the form
    $$p \to q$$
    $$\lnot q$$
    $$\text{therefore } \lnot p$$
\end{exmpl}

\begin{dfn}[label={def:generalization}]{Generalization}{dfnGeneralization}
    A \emph{generalization} is an argument of the form
    $$p.$$
    $$\text{Therefore } p \lor q.$$
\end{dfn}

\begin{exmpl}[label={exmpl:generalization}]{Generalization}{xmplGeneralization}
    An example of a \emph{generalization} is
    \begin{center}
        The sky is blue.\\
        Therefore, the sky is blue or grass is purple
    \end{center}
    Since the sky is blue, our conclusion here is true. It doesn't matter if grass is purple or not.
\end{exmpl}
\vspace{1cm}

The opposite of a generalization is called \emph{specialization}.

\begin{dfn}[label={def:specialization}]{Specialization}{dfnSpecialization}
    A \emph{specialization} is an argument of the form
    $$p \land q$$
    $$\text{Therefore } p$$
\end{dfn}

\begin{exmpl}[label={exmpl:specialization}]{Specialization}{xmplSpecialization}
    An example of a \emph{specialization} is
    \begin{center}
        The sky is blue and grass is green.\\
        Therefore, the sky is blue.
    \end{center}
    Since both of the conditions in our premis are true, then we can pull one of them out and say that is definitely true.
\end{exmpl}
\newpage

One final, but very important, form of an argument is a \emph{contradiction}, not to be confused with \Cref{def:contradiction}.

\begin{dfn}[label={def:contradictionArgument}]{Contradiction (Argument)}{dfnContradictionArgument}
    A \emph{contradiction} is an argument of the form
    $$\lnot p \to c$$
    $$\text{Therefore } p$$
    Where $c$ is the contradiction from \cref{def:contradiction}. In plain English this argument says ``Assume $p$ is false, this leads to a contradiction. Therefore $p$ must be true".
\end{dfn}
\vspace{1cm}
We shall look at an example of this later on.

\section{Analyzing an Argument for Validity}
Let's take a look at an argument and try analyzing it to see if the argument is valid.

\begin{exmpl}[label={exmpl:argumentValidity}]{Analysing an Argument for Validity}{xmplArgumentValidity}
    Let's look at the following argument\\
    \begin{center}
        If I'm skilled at poker, then I will win.\\

        I won money playing poker.\\

        Therefore, I'm skilled at poker
    \end{center}
    The first premise is of the form $p \to q$ where $p$: \emph{I'm skilled at poker} and $q$:\emph{I will win}.

    Using these statements, the full format of the argument is \\
    \begin{center}
        $p \to q$\\
        $q$\\
        Therefore $p$
    \end{center}

    This is not a valid argument form. There could be other reasons I won money at poker other than being skilled - it could have just been pure luck, or the other players may have been even worse at poker.\\

    This argument would be valid if the first premise instead was ``\emph{If I win, then I am skilled at poker}".
\end{exmpl}
